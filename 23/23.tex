\newpage
\section{Билет 23. Согласие в распределенной системе. Задача византийских генералов: распространение значения (agreement), согласование решения (consensus), согласование вектора (interactive consistency). Невозможность решения при трёх процессах и одном сбое. Алгоритм Лэмпорта для "устных" сообщений.}


\begin{center}
    \textit{\underline{Согласие в распределённой системе.}}
\end{center}

\textit{Постановка задачи:} \\
Задано N процессов, у каждого есть некие данные — предложение (proposal), они должны выполнить некоторый распределённый алгоритм и прийти к решению (decision).

\textit{Требования:} 
\begin{itemize}
\item Основные:
\begin{itemize}
\item \textbf{Согласие (agreement)}: все не отказавшие (не упавшие навсегда) процессы должны завершиться с решением (decide) и все эти решения должны совпадать.
\item \textbf{Нетривиальность (non-triviality)}: должны быть варианты исполнения, приводящие к разным решениям (возможно, просто с разными исходными предложениями или разным исходным состоянием процессов).
\end{itemize}
\item Дополнительные: 
\begin{itemize}
\item \textbf{завершение (termination)}: протокол должен завершиться за конечное время.
\end{itemize}
\end{itemize}


\begin{center}
    \hypertarget{Byzantine_fault} {\textit{\underline{Задача византийских генералов}}}
\end{center}

\textit{Формулировка: }

Византия. Ночь перед великим сражением с противником. Византийская армия состоит из N легионов, каждым из которых командует свой генерал. Также у армии есть главнокомандующий, которому подчиняются генералы. В то же самое время империя находится в упадке, и любой из генералов и даже главнокомандующий могут быть предателями Византии, заинтересованными в её поражении. Ночью каждый из генералов получает от главнокомандующего приказ, как надлежит поступить в 10 часов утра (время одинаковое для всех и известно заранее). Варианты приказа: «атаковать противника» или «отступать».

Возможные исходы сражения:
\begin{enumerate}
\item Если все верные генералы атакуют -- Византия уничтожит противника (благоприятный исход).
\item Если все верные генералы отступят -- Византия сохранит свою армию (промежуточный исход).
\item Если некоторые верные генералы атакуют, а некоторые отступят -- противник со временем по частям уничтожит всю армию Византии (неблагоприятный исход).
\end{enumerate}
   
Также следует учитывать, что если главнокомандующий -- предатель, то он может дать разным генералам противоположные приказы, чтобы обеспечить уничтожение армии. Следовательно, генералам надо учитывать такую возможность и не допускать несогласованных действий. Если же каждый генерал будет действовать полностью независимо от других (например, сделает случайный выбор), то вероятность благоприятного исхода весьма низка. Поэтому генералы нуждаются в обмене информацией между собой, чтобы прийти к единому решению. 


\begin{center}
    \hypertarget{Byzantine_fault} {\textit{\underline{Алгоритм Лэмпорта}}}
\end{center}

Алгоритм решения для частного случая, когда количество генералов ограничено и не может динамически изменяться, был предложен в 1982. Алгоритм сводит задачу для случая $m$ предателей среди $n$ генералов к случаю $m-1$ предателя.

Для случая $m=0$ алгоритм тривиален, поэтому проиллюстрируем его для случая $n=4$ и $m=1$. В этом случае алгоритм осуществляется в 4 шага.

\begin{enumerate}
\item Каждый генерал посылает всем остальным сообщение, в котором указывает численность своей армии. Лояльные генералы указывают истинное количество, а предатели могут указывать различные числа в разных сообщениях. Генерал 1 указал число 1 (одна тысяча воинов), генерал 2 указал число 2, генерал 3 (предатель) указал трём остальным генералам соответственно $x$, $y$, $z$ (пусть истинное значение – 3), а генерал 4 указал 4.

\item Каждый формирует свой вектор из имеющейся информации:
\begin{itemize}
\item Вектор генерала №1: $(1,2,x,4);$
\item Вектор генерала №2: $(1,2,y,4);$
\item Вектор генерала №3: $(1,2,3,4);$
\item Вектор генерала №4: $(1,2,z,4).$
\end{itemize}

\item Каждый посылает свой вектор всем остальным (генерал 3 посылает опять произвольные значения). После этого у каждого генерала есть по четыре вектора:

\begin{center}
\begin{tabular}{|c|c|c|c|c|}
\hline
      & $g_1$       & $g_2$       & $g_3$       & $g_4$       \\
\hline
$g_1$ & $(1,2,x,4)$ & $(1,2,x,4)$ & $(1,2,x,4)$ & $(1,2,x,4)$ \\
\hline
$g_2$ & $(1,2,y,4)$ & $(1,2,y,4)$ & $(1,2,y,4)$ & $(1,2,y,4)$ \\
\hline
$g_3$ & $(a,b,c,d)$ & $(e,f,g,h)$ & $(1,2,3,4)$ & $(i,j,k,l)$ \\
\hline
$g_4$ & $(1,2,z,4)$ & $(1,2,z,4)$ & $(1,2,z,4)$ & $(1,2,z,4)$ \\
\hline
\end{tabular}

\end{center}



\item Каждый генерал определяет для себя размер каждой армии. Чтобы определить размер $i$-й армии, каждый генерал берёт $(n-m)$ чисел -- размеры этой армии, пришедшие от всех командиров, кроме командира $i$-й армии. Если какое-то значение повторяется среди этих $(n-m)$ чисел как минимум $(n-m-1)$ раз, то оно помещается в результирующий вектор, иначе соответствующий элемент результирующего вектора помечается неизвестным (или нулём и т. п.).

Все лояльные генералы получают один вектор $(1,2,f(x,y,z),4)$ где $f(x,y,z)$ есть число, которое встречается как минимум два раза среди значений $(x,y,z)$, или «неизвестность»\ , если все три числа $(x,y,z)$ различны. Поскольку значения $x, y, z$ и функция $f$ у всех лояльных генералов одни и те же, то согласие достигнуто.
\end{enumerate}