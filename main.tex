\documentclass[specialist, subf, href, colorlinks=true, 12pt, times, mtpro, final]{disser}

\usepackage[utf8x]{inputenc}
\usepackage[english, russian]{babel}
\usepackage[T2A]{fontenc}
\usepackage{amsmath,amsthm,amssymb}
\usepackage {wrapfig}
\usepackage {enumitem}  
\usepackage{graphicx}
\usepackage{multicol}
\usepackage{mathrsfs}
\usepackage{xcolor}
\usepackage{hyperref}
\usepackage{tikz}
\usepackage{pdfpages}
\usepackage{algorithm}
\usepackage{algpseudocode}


\usetikzlibrary{decorations.pathreplacing}
\usepackage[a4paper, mag=1000, includefoot, left=1.5cm, right=1.5cm, top=1cm, bottom=1cm, headsep=1cm, footskip=1cm]{geometry}
\usepackage{tikz}
\newcommand{\RNumb}[1]{\uppercase\expandafter{\romannumeral #1\relax}}
\usetikzlibrary{graphs}

\theoremstyle{definition}
\newtheorem{defn}{Определение}[section]
\newtheorem{example}{Пример}[section]
\newtheorem{state}{Утверждение}[section]
\newtheorem{theorem}{Теорема}[section]
\newtheorem{lemma}{Лемма}[section]
\newtheorem{axiom}{Аксиома}[section]
\newtheorem{consequence}{Следствие}[section]

\definecolor{linkcolor}{HTML}{0000ff} % цвет ссылок
\definecolor{urlcolor}{HTML}{0000ff} % цвет гиперссылок
\hypersetup{pdfstartview=FitH, linkcolor=linkcolor,urlcolor=urlcolor, colorlinks=true}

\begin{document}
\begin{titlepage}
	\begin{center}
		
		Федеральное государственное бюджетное образовательное учреждение высшего образования 
		<<Московский Государственный Университет им.\,М.\,В.\,Ломоносова>>\\
		
		\vspace{9cm}
		Механико-математический факультет
		
		{\bf Конспект лекций по спецкурсу <<Введение в распределенные системы>>}
		
		\vspace{9cm}
		\begin{flushright}
			{\bfРаботу выполнили:}\\
			Кармацких Ирина, Дмитрий Сокунов, Всеволод Винницкий, Иван Федотов, Дмитрий Локтев, Валерия Лобанов, Николай Мартынов, Остап Карковский, Артур Сибгатуллин\\[0.5cm]
		\end{flushright}
		\vspace{1cm}
		
		\normalsize Москва, 2022
	\end{center}
\end{titlepage}

	
\tableofcontents
	
\newpage
\section{Билет 1. Реляционная база данных. Область применимости и основные понятия. Отношение, атрибут, кортеж. Связь. Целостность данных.}
\newpage
\section{Билет 2. Операции над данными. Объединение, пересечение и разность. Произведение и соединение по условию. Деление. Ограничение, проекция и переименование. Реляционное исчисление.}
\newpage
\section {Билет 3. Нормальные формы.}
\newpage
\section {Билет 4. Индексы (дерево, карты, хэш).}

Индекс применяется для ускорения поиска нужных строк (в операциях
выборки/обновления/удаления). Индекс является упорядоченной структурой (записи в индексе хранятся в отсортированном виде). После создания индекса в актуальном состояние его поддерживает СУБД.

По умолчанию команда CREATE INDEX создаёт индексы-B-деревья, эффективные в большинстве случаев. Выбрать другой тип можно, написав название типа индекса после ключевого слова USING. 

\textbf{B-дерева:} 

Структура B-дерева: 
\begin{itemize}
    \item При построении B-дерева применяется фактор t, который называется минимальной степенью. Каждый узел, кроме корневого, должен иметь, как минимум t – 1, и не более 2t – 1 ключей. Обозначается n[x] – количество ключей в узле x.
    \item Ключи в узле хранятся в неубывающем порядке. Если x не является листом, то он имеет n[x] + 1 детей. Если занумеровать ключи в узле x, как k[i], а детей c[i], то для любого ключа в поддереве с корнем c[i] (пусть k1), выполняется следующее неравенство $-k[i-1] \leq k1 \leq k[i]$ (для $c[0]: k[i-1] = -\infty,$ а для $c[n[x]]: k[i] = +\infty)$. Таким образом, ключи узла задают диапазон для ключей их детей.
    \item Все листья B-дерева должны быть расположены на одной высоте, которая и является высотой дерева. Высота B-дерева с $n \geq 1$ узлами и минимальной степенью $t\geq 2$ не превышает logt(n+1).
\end{itemize}

На рисунке \ref{fig:tree} изображен пример B-дерева. 

\begin{figure}[!h]
    \centering
    \includegraphics[scale = 0.5]{4\tree.jpg}
     \label{fig:tree}
    \caption{Пример B-дерева}
\end{figure}


B-деревья могут работать в условиях на равенство и в проверках диапазонов с данными, которые можно отсортировать в некотором порядке. Точнее, планировщик запросов PostgreSQL может задействовать индекс-B-дерево, когда индексируемый столбец участвует в сравнении 

\textit{Недостатки}: поиск только по одному столбцу. (Если нам нужно найти брюнетов с зелеными глазами, то по дереву мы найдем брюнетов, а дальше уже придется проверять их всех, то есть не использовать индексы)

\textbf{Bitmap(карты)} 

В bitmap-структурах создается двухмерный массив со столбцом для каждой строки в индексируемой таблице. Каждый столбец представляет отдельное значение в bitmap-индексе. Этот двухмерный массив показывает каждое значение индекса, умноженное на количество строк в этой таблице.

Идея поиска по такому индексу предсавленна на рисунке \ref{fig:Map}  

\begin{figure}[!h]
    \centering
    \includegraphics[scale = 0.3]{4\Map.jpeg}
     \label{fig:Map}
    \caption{Пример поиска по индексу}
\end{figure}


Берем строчку из первого индекса (по нашим условиям), и строчку из второго индекса. Дальше логические операции заменяем на побитовые и проводим их на наших выбранных строчках (Рис.\ref{fig:BMap}).

\begin{figure}[!h]
    \centering
    \includegraphics[scale = 0.3]{4\BitMat.png}
    \label{fig:BMap}
    \caption{Пример поиска по индексу}
\end{figure}

\textit{Недостатки:} Предположим, мы заблокировали запись -> нужно заблокировать часть индекса, где она учавствует (в дереве - только лист, там мало значений: k), а вот у карты нужно блокировать всю строку индекса - это много.


\textbf{Хеш - индексы} 
Хэш-индекс (Hash Index) основан на хэш-таблице, он полезен только для точного поиска по каждому столбцу в индексе. Просто по значениям в таблице вычисляем некоторую хеш функцию и значения таблицы помещаем в ячейку с номером хеша. 


\newpage
\section {Билет 5. Оптимизация запросов, построение и оценка планов запросов.}

Операции существуют 2-ух типов: 
\begin{itemize}
	\item DML - это аббревиатура от языка манипулирования данными. Он используется для извлечения, хранения, изменения, удаления, вставки и обновления данных в базе данных.
	
	Примеры: операторы SELECT, UPDATE, INSERT, DELETE, MERGE
	
	\item DDL - это аббревиатура языка определения данных. Он используется для создания и изменения структуры объектов базы данных в базе данных.
	
	Примеры: операторы CREATE, ALTER, DROP
\end{itemize}


SELECT - самый важный из DML

SELECT поля

FROM таблица 

WHERE условия 

Тут еще можно использовать GROUP BY и HAVING - подробно не обсуждали 
\\[10pt]
1)\textbf{Способы соединение таблиц }

Хотим понять каким способом делать соединения таблиц. Пусть хотим соединить 2 таблицы по id, те по A.id = B.id с условием A.a = x, B.b = y.

\textit{Способ 1}: \textbf{nested loop (вложенный цикл)}

Выбрать все данные из одной таблицы (например А по условию A.a= x) после этого для каждого id выбрать все подходящие из B (по индексу и значению B.b = y). По факту это 2 цикла вложенных в друг друга..

Сложность: $N^2$ или на другой лекции $N_A * M_{id}$  -  кол записей отфильтрованных в таблице A * сложность доступа к таблице M по индексу. Если критерий x является очень хороший, то сложность становиться N. 

\textit{Способ 2}: \textbf{merge join (слияние)}

Взять наши данные и отфильтровать отдельно по условию x в таблице А, по условию y в таблице В, а дальше отсортировать и после этого соединить два отсортированных массива.

Сложность: $NLogN$  или на другой лекции $NLogN * MLogM$. Если критерий x является очень хороший, а таблица В большая, то сложность останется такой же. 

Из-за этого нельзя сказать, какой способ (1 или 2) лучше. 

\textit{Способ 3}: \textbf{hash join (хеширование)}

При соединении хешированием строки одного набора помещаются в
хеш-таблицу, содержащуюся в памяти, а строки из второго набора
перебираются, и для каждой из них проверяется наличие
соответствующих строк в хеш-таблице.

Ключом хеш-таблицы является тот столбец, по которому выполняется
соединение наборов строк.

Такой способ эффективен для больших выборок. 
\\[15pt]
2) \textbf{Способы доступа к данным таблицы}

Теперь мы хотим выполнять запросы к одной таблице. У нас остается A.a= x. 

\textit{Способ 1}: \textbf{full scan} - проходить таблицу целиком. 

\textit{Способ 2}: \textbf{index} - используем индекс, если нужное значение заиндексировано и мы ищем какой-то диапазон. 

Пример: номер зачетки заиндексирован, и мы просто хотим найти номер зачетки, который равен x или находится в каком-то диапазоне. 

\textit{Способ 3}: \textbf{index scan} - предположим, что значение заиндексировано, а мы, например, хотим выбирать по значениям какой-то функции от значения в таблице, сам индекс такого не даст. Так что нам придется в любом случае читать данные. Вся запись весит много, а вот индекс, который хранит данные только одного значения, по факту весит гораздо меньше. 

Был пример с зачетками: заиндексирован номер зачетки, хотим получить человека с номером зачетки таким, что Sin от номера зачетки лежит в [0.5, 0.6], сам индекс такого не даст. Индекс хранить только номера зачеток, а полная запись содержит ФИО, место рождения, дату рождения и еще много всего. Соответственно, занимает много памяти и времени, если все читать из памяти для поиска, а в индексе мы читаем только номер зачетки, что весит гораздо меньше -> быстрее. 

\begin{figure}[H]
	\centering
	\includegraphics[scale = 0.5]{5/plan.jpg}
	\caption{}
	\label{fig:plan1}
\end{figure}

Пусть у нас есть запрос, в котором присутствуют таблицы A, B, C, D. Нужно построит план выполнения запроса, которое выглядит как дерево. (Это не синтаксический разбор!). Тут мы решили отсканировать A full scan, B по индексу, а потом сделать между ними hash join. Рис. \ref{fig:plan1}

Скорость выполнения ОЧЕНЬ сильно зависит от оптимальности плана запроса.


Для выбора плана запроса в базе данных существует \textbf{оптимизатор}. Он получает на вход разобранный синтаксически SELECT и выдает план выполнения запроса (внутри оптимизатора еще проверяются права доступа). 
\\[60pt]
\textbf{Виды оптимизаторов:} 
\textbf{Rule (cинтаксический)}. Основан на словарях, для того, чтобы выбрать лучший план запроса просматривает все условия, которые фигурируют в запросе и проводит анализ без учета данных (только на объявлении структуры). Например в нем прошито правило, что индексный поиск по уникальному индексу - наилучший, после него идет не уникальный индекс, потом индекс скан, а потом фулл скан. Правил около 15, но суть одна (по факту прописаны приоритеты). Работает это очень быстро, но наилучший план он выбрать не может.

Почему плохой оптимизатор: мы хотим найти в базе студентов Иванова Рамиля, тогда нам выгодно в первую очередь искать по имени, так как Ивановых скорее всего много. Этот оптимизатор такое не учитывает. 

\textbf{Cost} Строит много планов выполнения запроса и считает их стоимость. По факту делает "виртуальное" выполнение запроса. У него есть оценки сколько понадобится сделать действий, чтобы произвести какой-либо способ доступа. 

Например, если в оптимизаторе есть информация, что в таблице 1к блоков и в ней 1 миллион записей при этом уникальных 100к, тогда full scan должен прочитать все блоки и выбрать где-то 10 записей. Если же будет использоваться индекс, то нужно будет прочитать 10 блоков и выбрать 10 записей. 

Так для каждого плана появляется цена выполнения. Рассмотрим oracle: при расчете стоимости все операции (чтение диска, стоимость сортировки, процессорное время...) конвертируются в некоторую условную "валюту" , которая называется \textbf{логическим чтением блоков}. Стоимость высчитывается в этой валюте. Часто планы запроса хранятся, чтобы не пересчитываться много раз. 

Сложности: 

1) количество вариантов возможных планов запросов. Из-за этого оптимизатор работает классическими алгоритмами частичного обхода дерева (что-то вроде берутся наилучшие ветки, начинаем спуск, если оказалась сразу плохой, то мы ее откидываем и дальше не проверяем). И еще можно поставить ограничение на кол-во вариантов. 

2) Где брать данные, чтобы понимать, сколько мы получим записей после какой-то выборки. Если мы выбираем по фиксированным условиям (имя = Петр), то помогают строим гистограммы: по каждому столбцу строим общие характеристики и строим гистограммы сколько значений попадают в какой-то диапазон (на букву П у нас 8м записей, на букву К 10м и тд). Это частично решает проблему (так как строить уникальные гистограммы слишком затратно). 

Статистика не работает, если мы используем ограничения, которые невозможно заранее просчитать. Таких бывает 3 вида: 
\begin{enumerate}
	\item Для столбцов, которые имеют близкие значения с очень разными частотами.
	\item использование ограничений, которые содержат функцию (хотим выбрать студентов у которых третий знак после запятой в синусе от веса студента равен 3, понятно, что гистограмма бесполезна) 
	\item (самый важный для нас) Если мы имеем дело с доступом к удаленным данным: получаем данные с другого сервера, но статистики его мы не знаем.
\end{enumerate}

Есть два подхода к решению таких сложностей: 

\begin{enumerate}
	\item Семплирование - получение примера данных. (Например в oracle: A SAMPLE (5) - хотим считать из таблицы 5\% случайных блоков). Берутся примеры данных из всех таблиц, и на них проверяются условия выборки. Потом расширяем полученные данные на всю таблицу и считаем, что получили всю статистику. (для удаленных данных просто просим прислать какой-то процент блоков и далее аналогично). Способ хороший так как дает +- точные результаты, но есть недостаток - стоимость подготовки плана (может присылаться очень много данных).
	\item Перестраивание дерева на ходу. Реализуется сложно, но принцип простой:  смотрим на картинку плана запроса. Мы предполагаем, что при объединении (HJ) А и В мы получим примерно 4 записи (не стоимость) и тогда выше стоящий NL оптимальный вариант. Начали выполнять, в момент соединения А и В HJ мы получили 400к записей и в таком случае система останавливает выполнение запроса и производит пересчет кусков, которые мы еще не выполнили.
\end{enumerate}

Существует еще один метод оптимизации запроса - \textbf{переписывание}. 

Вспоминаем три структуры данных: \textit{таблицы} - просто хранят данные, \textit{представления} - запросы к таблицам (возможно, в виде плана запроса), \textit{материальные представления} - результат выполнения какого-то запроса. 

Нас интересуют материальные представления. У мп есть метка вычисления (мы можем сказать, когда менялись таблицы и когда было посчитано материальное представление). Смотрим на рисунок плана. Пусть мы захотели отдельно посмотреть HJ для A и В. Если HJ был сделан после изменения таблиц, то мы можем удалить часть дерева ниже HJ и подставить туда материальное представление ( Рис.\ref{fig:plan2}). Так мы сокращаем размер вычислений.  

\begin{figure}[H]
	\centering
	\includegraphics[scale = 0.5]{5/plan2.jpg}
	\caption{после добавления МП}
	\label{fig:plan2}
\end{figure}

\color{blue} Дальше говорим про удаленное выполнение запроса, не знаю, надо ли, но в билете нет. Если интересно 3 семинар, 29 минута. Тут напишу краткий список. 
\color{black}

Механизмы получения данных из внешней системы (терминология  oracle)
\begin{itemize}
	\item  dblink -  связывают бд одного типа (умеют все делать так как все друг о друге знают)
	\item внешние таблицы - если надо подключить статические данные (например csv файл). Может быть медленным, особо ничего нет
	\item шлюзы (ODBC) 
\end{itemize}
\newpage
\section {Билет 6. Транзакции, блокировки, версионность.}
\newpage
\section {Билет 7. Распределенные запросы. Распределенные и кластерные базы данных. Распределенные транзакции.}

1) Про распределённые транзакции.

Транзакция - это набор связанных задач, который, помимо всего прочего, завершается успешно (фиксация) или с ошибкой (отмена) как единое целое. \href{https://docs.microsoft.com/ru-ru/dotnet/framework/data/adonet/distributed-transactions}{Распределенная транзакция} — это транзакция, затрагивающая несколько ресурсов. Для фиксации распределенной транзакции все участники должны гарантировать, что любое изменение данных будет постоянным. Изменения должны сохраняться даже в случае фатального сбоя системы или других непредвиденных событий. Если хоть один из участников не сможет предоставить такую гарантию, вся транзакция завершится с ошибкой и будет выполнен откат любых изменений данных внутри области транзакции.

\bigskip
2) Про распределённые базы данных (про кластерные лучше просто прочесть \href{https://www.jetinfo.ru/klasternye/#gl_2_1}{здесь}).

\href{https://ru.wikipedia.org/wiki/%D0%A0%D0%B0%D1%81%D0%BF%D1%80%D0%B5%D0%B4%D0%B5%D0%BB%D1%91%D0%BD%D0%BD%D0%B0%D1%8F_%D0%B1%D0%B0%D0%B7%D0%B0_%D0%B4%D0%B0%D0%BD%D0%BD%D1%8B%D1%85}{Распределённая БД} - это БД, составные части которой размещаются в различных узлах компьютерной сети в соответствии с каким-либо критерием.

Распределённая база данных — это именно единая база данных, а не произвольный набор файлов, индивидуально хранимых на разных узлах сети и являющейся распределенной файловой системой. Данные представляют собой РБД, только если они связаны в соответствии с некоторым структурным формализмом, реляционной моделью, а доступ к ним обеспечивается единым высокоуровневым интерфейсом.

Распределённые базы могут иметь разный уровень реплицированности — от полного отсутствия дублирования информации, до полного дублирования всей информации во всех распределённых копиях (например, блокчейн).

Распределение (включая фрагментацию и репликацию) базы данных по множеству узлов невидимо для пользователей. Это свойство называется прозрачностью, а технология распределения и реплицирования данных по множеству компьютеров, связанных сетью, является основополагающей для реализации концепции независимости данных от среды хранения. Это обеспечивается за счёт нескольких видов прозрачности:

\begin{enumerate}
	\item[\textbullet] прозрачность сети, а следовательно, прозрачность распределения
	\item[\textbullet] прозрачность репликации
	\item[\textbullet] прозрачность фрагментации
	\item[\textbullet] прозрачность доступа, означающая, что пользователи имеют дело с единым логическим образом базы данных и осуществляют доступ к распределенным данным точно так же, как если бы они хранились централизованно.
\end{enumerate}

В идеале полная прозрачность подразумевает наличие языка запросов к распределённой СУБД, не отличающегося от языка для централизованной СУБД.

\bigskip
3) Про распределённые запросы лучше просто почитать \href{https://www.ibm.com/docs/ru/informix-servers/12.10?topic=SSGU8G_12.1.0/com.ibm.admin.doc/ids_admin_0069.htm}{здесь} и \href{https://studfile.net/preview/9032573/page:10/}{вот здесь}.

\newpage
\section {Билет 8. Одно-, двух- и трехзвенные архитектуры.}
\newpage
\section {Билет 9. IP, TCP, UDP, VPN, NAT}
\newpage
\section {Билет 10. Методы обеспечения отказоустойчивости. RAID, распределенное хранение данных, виды и репликаций.}
\newpage
\section {Билет 11. Метода шифрования данных.}
\newpage
\section {Билет 12. Полнотекстовый поиск. Алгоритмы без индексов. Поиск с реверсивным индексом.}

Текстом назовем произвольный упорядоченный набор слов в некотором алфавите. Фрагмент текста или просто фрагмент — поднабор текста, в который входят подряд идущие слова с сохранением порядка.
\href{https://ru.wikipedia.org/wiki/Полнотекстовый_поиск}{Полнотекстовый поиск} - автоматизированный поиск документов, при котором поиск ведётся не по именам документов, а по их содержимому, всему или существенной части.

\subsection{Безиндексный поиск}
\href{https://habr.com/ru/company/vk/blog/270507/}{regex}

\subsection {Реверсивный индекс}
Пусть у нас выборка текстов $T_1, T_2, \dots T_{N}$, занумерованные натуральными числами. Есть несколько ключевых слов $k_1, k_2, \dots k_n$ и есть файлы $F_1, \dots F_n$. В файле $F_i$ записаны номера текстов, в которых встречается слово $k_i$. Приведем пример, пусть есть два слова СТОЛ, СТУЛ, и файлы $F_1, F_2$  имеют вид: \\
$$ F_1: 1,3,4,7,9 $$
$$ F_2: 2,3,4,5,6,8 $$

Пусть номера текстов в файлах отсортированы!
Тогда алгоритм поиска прост: мы производим сливание отсортированных массивов (соответственно, если нам нужно найти слова СТОЛ И СТУЛ, то выбираем файлы в которые входят оба эти слова, если  СТОЛ ИЛИ СТУЛ, то просто сливаем два массива). Так как массивы в файлах отсортированы, то задача слияния выполняется быстро.\\
Вопрос: если один файл изменили, что делать ? Например в 1 тексте слово ТАБУРЕТКА заменили на СТУЛ. Текст 1 попадает в специальный файл где хранятся удаленные тексты, а номер 1 заменяется, например, на 10. Тогда:
$$ F_1: 3,4,7,9, 10 $$
$$ F_2: 2,3,4,5,6,8,10 $$
$$DEL (\text{удаленные}): 1$$

Если в 4 файле заменили, слово СТУЛ на ТАБУРЕТКА, то (номер 4 заменяем на 11) имеем:
$$ F_1: 3,7,9, 10, 11 $$
$$ F_2: 2,3,5,6,8,10 $$
$$DEL (\text{удаленные}): 1, 4$$

Когда все номера становятся очень большими их все можно уменьшить.
\newpage
\section {Билет 13. Обработка текстов. Морфологический анализ, tf*idf, ранжирование, исправление опечаток, классификация, кластеризация, поиск дубликатов.}

Пусть у нас есть пачка текстов $T_1, T_2, T_3 \dots T_n$, которую мы будем исследовать \\
Методы представления текстов
\begin{itemize}
\item множетсвенный. Текст представляется как множество слов, для которых не определено отношение порядка. Преимущество - простота. Для вычисления сходства двух текстов используется простая формула $k = \frac {|A \cup B|}{|A \cap B|}$. Если коэффициент большой, то тексты похожи друг на друга.
\item векторная модель. Пусть есть выделенные  "базисные" слова $w_1, w_2, \dots w_n$, тогда текст представляется вектором в n-мерном пространстве, где, в простейшем случае, координаты вектора строятся так: i - ая координата это количество вхождения слова $w_i$ в текст. Но на практике используют другой способ определения координат (способ с помощью модели tf*idf)  
\item линейная модель (про нее не говорили)
\item семантическая (очень сложная и мы ее особо не обсуждали).
\end{itemize}

\subsubsection {Модель tf*idf для построения векторов}
\href{https://ru.wikipedia.org/wiki/Векторная_модель}{Векторная модель} \\
\underline {Определние}
tf для слова $w_i$ в тексте $T_j$ определяется так: 
 $tf = log (Q)$, Q - сколько раз $w_i$ встретилось в $T_j$.  \\
idf для слова $w_i$ в тексте $T_j$ определяется так: 
$idf = \frac{1}{log(P)}$, P - количество документов в последовательности $T_1, T_2, T_3 \dots T_n$, в которых встречается слово $w_i$ \\
На практике используют именно произведение $tf*idf$ - означающее важность слова в документе. Произведение tf*idf отражает закон Зипфа:
На графике по вертикальной оси откладывается важность (т.е. tf*idf) по горизонтальной частота встречаемости слова в конкретном документе.
Выделяется три зоны: \\
1 - редкие слова, слова придуманные автором или опечатки \\
2 - ключевые слова, определяющие текст \\
3 - обычные слова связки \\

\includegraphics[width=0.5\linewidth]{13/Zipf}
Поэтому, в итоге, текст это вектор, где на i - ом месте стоит tf*idf слова $w_i$. Мера близости текстов A и B в векторной модели - косинус угла между векторами A и B.

\subsubsection {Морфологический анализ}
\href{http://prutzkow.com/ru-ru/science/natural-language-processing/morphology/}{Морфологический анализ} \\
Перед построением векторов сначала несколько унифицируют используемые слова, для этого используют морфологический анализ.
Морфологический анализ может быть словарным (со словарем основ и окончаний или словарем словоформ) или бессловарным (только со словарем окончаний; словарь окончаний может быть встроен в алгоритм морфологического анализа). Бессловарный метод используется только для определения переменной морфологической информации (не всегда однозначно), а словарный — во всех остальных случаях. 

\subsubsection {Исправление опечаток}
\href{https://ru.wikipedia.org/wiki/Расстояние_Левенштейна}{Расстояние по Левенштейну} \\
Элементарная операция по Левенштейну - это добавление буквы, удаление буквы или замена одной буквы на другую. Расстояние это количество операций необходимое для того, чтобы из одного слова получить другое с помощью операций Левенштейна. Например, слова aabb, aabbb, aab, aacb находятся на одинаковом расстоянии (расстояние равно 1). Далее, чтобы исправить опечатку необходимо найти ближайшее слово по Левенштейну.

\subsubsection {Поиск дубликатов}
Если необходимо решить задачу на полное совпадение текстов, то она делается очень просто. Для каждого документа берется хэш функция. С большой вероятностью документы с совпадающем хэшем - совпадают. Но на практике необходимо искать не в точности совпадающие тексты, а просто похожие. В таком случае используют понятие N-грамм.  \\
Определение: N-грамма — последовательность из n элементов. Например для текста: Мама мыла раму, все 4 граммы имеют вид: мама, амам, мамы, амыл, мыла, ылар, лара, арам, раму.  \\
Теорема. Если документы почти совпадают, то и N-граммы почти совпадают\\
Это отличный способ, сравнивать все N-граммы, но есть одна проблема - их очень много и сравнение всех N-грамм двух текстов займет много времени. Поэтому действуют следующим образом

\begin {itemize}
\item Берутся все N-граммы двух текстов и к ним применяется какой-нибудь хеш. (хэш функция от N - граммы называется называется шинглом)
\item Береться 84 шингла (значения хэш функции), они объединяются в 12 групп
\item На каждой группе шинглов опять вычисляется хэш функция. (значение хэш функции на группе шинглов называется супершинглом)
\item Если у двух текстов совпадает хотя бы один супершингл,  то они потенциально совпадают.
\end {itemize}

\subsubsection {Классификация и кластеризация}
Классификация - разделение на текстов группы, которые мы заранее определили. Кластеризация - разделение на группы, заранее не известные.
Примеры разделение: по авторству, по эмоциональной окраске текста, по цели (повествование/реклама), по различным темам.
Методы (одинаково применяются для кластеризации и классификации)

\begin {itemize}
\item Тезаурас. Берется двудольный граф, в котором в одной доле (будем ее называть доля  W) располагаются термины, а во второй доле - темы (будем ее называть T). Вершина из W соединяется с вершиной из T, когда соответсвующий термин описывает данную тему. Данный граф дается нам сверху лингвистами или еще кем то. Далее мы читаем полученный текст и считаем сколько раз встретился тот или иной термин. Термин, наиболее часто встречающийся, выбирается победителем и тема к которой он относится - тема текста. Минус этого метода: кто-то должен нам дать двудольный граф.
\item Самообучающиеся карты.
Пусть стоит задача разделить цветные точки на группы так, чтобы в каждой группе были точки одного цвета. В начальный момент времени все точки расположены в трехмерном пространстве хаотично (Почему трехмерном? Каждый цвет однозначно представляется тремя числами в системе rdb). Выбираются функции $f_1, f_2, f_3$ как показано на рисунке (Интеграл от них равен 1)
Далее используется следующий алгоритм:
\begin {itemize}
\item Для точки A в ее окрестности считается сумма $S = \sum_{p_i \in U} c (p_i) f_1 (\rho_i)$, где $U$ - окрестность точки A, $p_i$ - точки, принадлежащие этой окрестности, $c (p_i)$ - цвет точки $p_i$ (вектор в трехмерном пространстве), $\rho_i$ - расстояние от точки $p_i$ до точки A. Точка А красится в цвет S. Так последовательно делаем для всех точек.
\item Проделываем процедуру еще раз с функциями $f_2, f_3$
\item В итоге Все точки разделяться на группы.
\end {itemize}

\includegraphics[width=0.5\linewidth]{13/func}

\begin{figure}[h!]
\begin{minipage}[h]{0.49\linewidth}
\center{\includegraphics[width=\linewidth]{13/begin}}
\end{minipage}
\hfill
\begin{minipage}[h]{0.49\linewidth}
\center{\includegraphics[width=\linewidth]{13/result}}
\end{minipage}
\end{figure}

\item  Физическая расстановка.
Стоит задача разделит точки на группы. Припишем каждой точке свою массу. Зададим векторы силы всемирного тяготения по формуле 
$\vec{F_{A,B}} = G \frac{m_{A} m_{B}\vec{r}}{|\vec{r}|^3}$, где $\vec{r}$ - вектор соединяющий точки А и В, $m_{A} m_{B}$ - массы точек. Тогда за время $\Delta t$ каждая точка под действием равнодейтсвующей силы притяжения передвинется на расстояние $\Delta s_i$ (свое для каждой точки). Если точки находятся близко, то объединяем их в одну точку суммарной массы. 

\item еще есть геометрическая расстановка \href{https://habr.com/ru/post/101338/}{habr.com}
\end{itemize} 



\subsubsection {Ранжирование}
Зачастую на запрос пользователя подходит несколько документов. Как выбрать один, наиболее подходящий? Для этого считают функцию ранжирования: $R = S A$, $S$ - функция соответствия запросу, A - функция авторитетности. 
\begin{itemize}
\item $S$ - функция соответствия запросу. Как обсуждалось выше это может быть косинус угла между текстом и запросом или например \href{https://ru.wikipedia.org/wiki/Okapi_BM25}{BM25}
\item A - функция авторитетности конкретного документа (эта функция своя для каждого документа и ее значение не зависит от запроса). Приведем пример определения авторитетности: Пусть есть ссылочный граф, т.е. в вершинах графа тексты и если текст A ссылается на текст B, то в графе есть соответсвующее ребро. Запустим в граф робота, который будет ходить по графу по следующему правилу, с вероятность p робот переходит из вершины в которой находится в одну из вершин на которую ссылается данная, и с вероятность (1 - р) - переходит в случайную вершину графа. Чем больше раз робот прошел через вершину, тем больше ее авторитетность.
\end {itemize}




\newpage
\section{Билет 14. Свойства распределенных систем: конкурентность, отсутствие состояния, независимые сбои. Основные проблемы и аспекты проектирования: неоднородность, открытость, безопасность, масштабируемость, отказоустойчивость, конкурентность, прозрачность.}

Существует множество определений распределенной системы, причем все они не является строгими или общепринятыми.

Приведём одно из них:\\
Распределенная  система – это набор  независимых  компьютеров,  не имеющих общей совместно используемой памяти и общего единого времени(таймера) и взаимодействующих через коммуникационную сеть  посредством  передачи  сообщений,  где  каждый  компьютер использует  свою  собственную  оперативную  память  и  на  котором выполняется  отдельный  экземпляр  своей  операционной  системы. Однако,  эти  операционные  системы  функционируют  совместно, предоставляя свои службы друг другу для решения общей задачи.\\

\textbf{ Свойства распределённой системы:}
\begin{enumerate}
\item \textbf{Конкурентность}

Процессы в системе работают параллельно, т.е. одновременно происходит несколько событий. Другими словами, компьютеры сети выполняют свои задачи одновременно, но независимо друг от друга. Следовательно, необходима их координация.

Для того, чтобы распределённая система работала, необходимо иметь способ определения порядка, в каком происходят события. Однако, когда несколько компьютеров работают одновременно, зачастую невозможно сказать, какое из событий произошло раньше, а какое позже, поскольку компьютеры пространственно разнесены. Другими словами, нет единых глобальных часов, которые определяют последовательность событий, происходящих на компьютерах сети.

\textit{Показательный пример этого свойства - \textcolor{blue}{\href{https://clck.ru/puqyK}{задача об обедающих философах.}}}

\item \textbf{ Отсутствие состояния}

Поскольку в распределённых системах нет понятия совместного используемой памяти, часто трудно определить текущее состояние системы. Определение глобального состояния распределённой системы можно выполнить путём синхронизации всех процессов так, чтобы каждый из них определил и сохранил своё локально состояние вместе с сообщениями, которые передаются в этот момент. Сама по себе синхронизация может быть выполнена без остановки процессов и записи их состояния. Вместо этого в ходе работы распределённой системы с неё можно сделать мгновенный слепок - распределённой снимок состояния.

\item \textbf{ Независимые сбои}

Характерной чертой распределённых систем, которая отличает их от единичных компьютеров, является устойчивость к частичным отказам, т.е. система продолжает функционировать после частичных отказов.

\textit{Показательный пример этого свойства - \textcolor{blue}{\href{https://clck.ru/JVS7z}{задача византийских генералов}} (также описана в билете \hyperlink{Byzantine_fault}{23})}\\
\end{enumerate}

\noindent При разработке или проектировании распределённой системы, обычно возникают следующие проблемы:
\begin{enumerate}
\setlength\itemsep{0.32em}
\item
\textbf{ Неоднородность} той среды, в которой, разворачивается система (вычислительные устройства, операционные системы, языки программирования, реализации однотипного ПО). 

В распределённую систему могут входить совершенно разные компоненты. Что вызывает трудности при их взаимодействии.
К примеру, значительная часть усилий программирования направленна на то, чтобы создать видимость простоты (Объектно-ориентированное проектирование).
То есть неоднородные элементы стараются спрятать за унифицированным интерфейсом.

\item
\textbf{ Открытость} - способность системы к расширению, открытые интерфейсы. 

Открытые приложения - это приложения, с которыми можно общаться по общеизвестному протоколу. Например в протоколе http есть общедоступная спецификация, поэтому, кем и как реализован клиент и сервер, неважно, а протокол общения с базой данных oracle не опубликован и для связи с ней нужно клиентское ПО.

\item
\textbf{ Безопасность} - политика безопасности, сопоставление правил подсистем, отказы в обслуживании, безопасность данных.

В различных организациях существуют разные правила по работе с данными, которые нужно как-то объединить. \\
DDos атаки также считают одной из угроз безопасности, поэтому у системы должен быть предусмотрен механизм отказа в обслуживании, для обеспечения её высокодоступности.\\
А также необходимо учесть многие вопросы, связанные с безопасностью данных, например, чтобы данные нельзя было заменить неконтролируемым образом, безопасность передачи, конфиденциальность данных и т.д.

\item
\textbf{ Масштабируемость} - стоимость оборудования, производительность ПО, анализ "узких"\ мест.

В общем случае масштабируемость определяют как способность вычислительной системы эффективно справляться с увеличением числа пользователей или поддерживаемых ресурсов без потери производительности и без увеличения административной нагрузки на ее управление. При этом систему называют масштабируемой, если она способна увеличивать свою производительность при добавлении новых аппаратных средств.  

\item
\textbf{ Отказоустойчивость} - обнаружение и устранение ошибок, управление сбоями.

В распределённой системе всё может пойти не так.
Многие сбои хочется устранять независимо для конечного пользователя.

\item
\textbf{ Конкурентность} - непротиворечивость данных.

Совместная работа пользователей тоже является проблемой.
Например заказ авиабилетов. Есть единая система бронирования, а точек, где можно продать билет много.

\item
\textbf{ Прозрачность}

\begin{itemize}
\item
\textit{\textbf{ Прозрачность доступа.}} (Локальные и удаленные)

Прозрачность в данном случае заключается в обеспечении сокрытия различий доступа и предоставлении данных.

\item
\textit{ \textbf{Прозрачность местоположения. }}

В распределенных системах прозрачность местоположения заключается в том, что пользователь не должен знать, где расположены необходимые ему ресурсы. 
Файлы могут перемещаться на различные узлы распределённой системы, но при этом, пользователь не должен замечать эти перемещения.

\item
\textit{\textbf{ Прозрачность совместной работы.}}

Различные пользователи распределенных систем должны иметь возможность параллельного доступа к общим данным. При этом необходимо обеспечить параллельное совместное использование ресурсов системы, а соответственно, обеспечить сокрытие факта совместного использования ресурсов.

\item
\textit{\textbf{ Прозрачность репликации. }}

В целях обеспечения сохранности данных, особенно на распределенных файловых системах, необходимо обеспечить репликацию данных. Пользователю не должно быть известно, что репликация данных существует.

\item
\textit{\textbf{ Прозрачность восстановления после сбоев.}}

Возможность восстановления данных, если что-то пошло не так, .

\item
\textit{\textbf{ Прозрачность масштабировния.}}

Масштабируемость распределённой системы также не должна быть заметна конечному пользователю. До недавнего времени основной подход, позволяющий расширять систему, заключался в наращивании различных ресурсов системы, к примеру, оперативной памяти, количества и объема жестких дисков. 
\end{itemize}

\end{enumerate}

\newpage
\section {Билет 15. Модели взаимодействия, ошибок, безопасности.}

Различные архитектуры распределённых систем имеют много общего.\\
\\
\textbf{1. Модели взаимодействия}
\begin{itemize}
\item Характеристики коммуникационного канала:

Основные характеристики это:

\hspace{-6px}\textit{Пропускная способность} - сколько гбит можно передать за единицу времени.

\hspace{-6px}\textit{Задержка} - сколько времени уходит на передачу сообщения, зависит от физической структуры и количества промежуточных структур.

\hspace{-6px}\textit{jitter} - изменение величины задержки при передаче последовательности пакетов, то есть существует некоторая средняя задержка, но каждый конкретный пакет передается быстрее или медленнее. Каким-то приложениям оптимизация jitter важна, каким-то нет, например при скачивании web страницы, он не важен, а в мультимедийном приложении будет важен. Один из вариантов оптимизации jitter-а в случае показа видео, может быть допущение потери пакетов, так как небольшую их потерю пользователь не заметит, а задержки во время показа, более ощутимы и проблематичны.

\item Упорядочивание сообщений (причинно-следственный порядок).

Мы можем спокойно сравнивать время события в одном процессе, но между разными процессами их можно сравнивать только при наличии передачи сообщения, используя то, что нельзя получить сообщение раньше, чем оно было отправлено.

\includegraphics[scale=0.7]{15/masseg_time.pdf}

Так, на картинке, процесс 2 знает порядок временных меток в процессе 1 до передачи сообщения (1 и 3) и знает что они все были до метки 4, но сравнить порядок 1 или 3 с меткой 2 нельзя, так в распределённой системе отсутствуют глобальные часы и синхронизировать раз и на всегда процессы нельзя.
 
 Более подробно этот пункт разобран в билете 20.
 
\item  Синхронные и асинхронные системы.

\hspace{12px}В \textit{синхронной системе} предполагается, что каждое сообщение будет доставлено в течение некоторого заданного времени (если пакет не был доставлен за какое-то время, то он не будет доставлен никогда), а также известно время ответа любого другого узла (время выполнения операций).
Это позволяет пользователям моделировать протокол с фиксированной верхней границей, т.е. со значением максимального времени, требуемого на доставку сообщения принимающей стороне.

\hspace{12px}В \textit{асинхронной системе} предполагается, что сеть может произвольно задерживать сообщения на любой период времени, дублировать их или менять их порядок. Другими словами, фиксированная верхняя граница времени, необходимого на отправку и получение сообщения, отсутствует.

\hspace{12px} Асинхронная система менее удобная, но более распространённая в реальных распределённых системах, так как обычно компьютеры могут выходить из строя или уходить в автономный режим, а сообщения могут удаляться, дублироваться, задерживаться или поступать в произвольном порядке из-за проблем сети.
\end{itemize} 
\textbf{\noindent 2. Модели ошибок}

Отказ какого-то процесса может быть двух типов:
\begin{itemize}
\setlength\itemsep{0.0001em}
\item \textit{Наблюдаемый отказ}  -  другие участники могут узнать, что он погиб.
\item \textit{Ненаблюдаемый отказ}  - другие участники не могут узнать, что он погиб. Такие отказы более распространённые, поэтому, когда в алгоритме нужно найти гарантированно "живой" процесс, его часто ищут на основе таймеров, то есть если какой-то процесс не отвечает, то он начинает подозреваться в отказе.
\end{itemize}

Также ошибки могут быть следующих типов:\\
\textit{Потеря} (в канале) - сообщение отправлено, но не дошло. \\
\textit{Пропуск отправки} (в процессе) - процесс не отправляет сообщения.\\
\textit{Пропуск приема} (в процессе) - процесс не получает сообщения.\\
\textit{Произвольное или византийское поведение} (и в канале, и в процессе) - поведение, не предписанное протоколом, возникшее из-за ошибок или же злоумышленно.\\
\\
\newpage
\textbf{\noindent 3. Модели безопасности}

Политика информационной безопасности - документ на естественном языке

Модель информационной безопасности - формализация этой политики, то есть выражение утверждений, написанных на русском языке, в терминах какого-то формального языка, а также реализация этой модели в распределённой системе. \\

В модели безопасности должны быть определенны:
\begin{itemize}
\item Описание информационной системы
\item Структурно-функциональные характеристики
\item Описание угроз безопасности
\item Модель нарушителя
\item Возможные уязвимости
\item Способы реализации угроз
\item Последствия от нарушения свойств безопасности информации.
\end{itemize}





\newpage
\section {Билет 16. Архитектуры информационных систем: "монолитная", клиент-серверная, многоуровневая. Одноранговые системы (peer-to-peer).}
\newpage
\section{Билет 17. Основные уровни сетевых протоколов. Семиуровневая модель OSI. Сети пакетной коммутации. Маршрутизация сообщений.}\label{b17:part1}

\textcolor{olive}{Клик сюда~\ref{b17:part2} к «Сети пакетной коммутации. Маршрутизация сообщений.»}

\textbf{Протокол} -- это набор соглашений между разработчиками ПО и аппаратуры. Текст протокола отвечает на вопрос: «Что нужно сделать, чтобы программы и устройства могли взаимодействовать с другими программами/устройствами, поддерживающими протокол».

\textbf{OSI} (Open Systems Interconnection model) -- открытая сетевая модель.

Модель описывает, какие уровни должны быть в сети и какие функции выполняются на каждом из уровней. OSI модель разделяет все протоколы на 7 таких уровней:
\begin{enumerate}
\item Физический (Physical)
\item Канальный (Datalink)
\item Сетевой (Network)
\item Транспортный (Transport)
\item Сеансовый (Session)
\item Представительный (Presentation)
\item Прикладной (Application)
\end{enumerate}

Все они отвечают за определенную ступень процесса отправки сетевого сообщения, а также содержат в себе определенную информацию. Шаги выполняются, обособленно друг от друга. Модель OSI \textbf{не включает} описание протоколов; они определяются в отдельных стандартах.

\textbf{Media layers} (уровни 1-3) управляют физической доставкой данных по сети. \\
\textbf{Host layers} (уровни 4-7) способствуют обеспечению точной доставки данных между компьютерами в сети.

\textbf{PDU} (Protocol data unit) -- это информация, доставляемая между одноуровневыми объектами сетей.

\begin{figure}[H] \centering
	\includegraphics[scale = 0.6]{17/OSI.png}
	\caption{Уровни модели OSI}
\end{figure}

\textbf{Физический уровень} отвечает за обмен физическими сигналами между физическими устройствами. Основная его цель представить нуль и единицу в качестве сигналов, передаваемые по среде передачи данных. Никакой гарантии корректности передачи нет.

\textbf{Канальный уровень} предназначен для обеспечения взаимодействия сетей на физическом уровне и контроля ошибок, которые могут возникнуть. Канальный уровень получает (от физического уровня) биты, находит начало и конец сообщения и упаковывает биты в кадры (фреймы). проверяет их на целостность и, если нужно, исправляет ошибки (либо формирует повторный запрос повреждённого кадра) и отправляет на сетевой уровень. Задача здесь — сформировать кадры с адресом отправителя и получателя, после чего отправить их по сети.

У канального уровня есть два подуровня -- это MAC и LLC. MAC (Media Access Control, контроль доступа к среде) отвечает за присвоение физических MAC-адресов, а LLC (Logical Link Control, контроль логической связи) занимается проверкой и исправлением данных, управляет их передачей.

\textbf{Сетевой уровень} предназначен для определения пути передачи данных. Получает MAC-адрес от коммутаторов с предыдущего уровня и занимаются построением маршрута от одного устройства к другому с учетом всех потенциальных неполадок в сети. Преобразует физический адрес (MAC) в логический (IP).

\textbf{Транспортный уровень} предназначен для обеспечения надёжной передачи данных от отправителя к получателю (т.е. без потерь).

При передаче по протоколу TCP, данные делятся на сегменты. Сегмент -- это часть пакета. Когда приходит пакет данных, который превышает пропускную способность сети, пакет делится на сегменты допустимого размера. Сегментация пакетов также требуется в ненадежных сетях, когда существует большая вероятность того, что большой пакет будет потерян или отправлен не тому адресату.

При передаче данных по протоколу UDP, пакеты данных делятся уже на датаграммы. Датаграмма (datagram) -- это тоже часть пакета, но ее нельзя путать с сегментом. Главное отличие датаграмм в автономности. Каждая датаграмма содержит все необходимые заголовки, чтобы дойти до конечного адресата, поэтому они не зависят от сети, могут доставляться разными маршрутами и в разном порядке.

Датаграмма и сегмент -- это два PDU транспортного уровня модели OSI. При потере датаграмм или сегментов получаются «битые» куски данных, которые не получится корректно обработать.

\textbf{Сеансовый уровень} устанавливает и поддерживает продолжительные соединения, сеансы связи.
Уровень управляет созданием/завершением сеанса, обменом информацией, синхронизацией задач, определением права на передачу данных и поддержанием сеанса в периоды неактивности приложений.
Примером работы пятого уровня может служить видеозвонок по сети. Во время видеосвязи необходимо, чтобы два потока данных (аудио и видео) шли синхронно.

\textbf{Представительный уровень} занимается тем, что представляет данные (которые все еще являются PDU) в понятном человеку и машине виде. Например, когда одно устройство умеет отображать текст только в кодировке ASCII, а другое только в UTF-8, перевод текста из одной кодировки в другую происходит на шестом уровне.
Шестой уровень также занимается представлением картинок (в JPEG, GIF и т.д.), а также видео-аудио (в MPEG, QuickTime). Помимо перечисленного, шестой уровень занимается шифрованием данных, когда при передаче их необходимо защитить.

\textbf{Прикладной уровень} -- это ближайший уровень к пользователю. На нем реализуются протоколы на уровне приложений, такие как HTTP.  Его задача, визуализировать или записать данные, взаимодействовать с пользователем.

\textbf{Инкапсуляция} -- весь процесс преобразования данных (с верхнего уровня) в сигналы (на нижний уровень), обратный ему процесс называется \textbf{декапсуляцией}.

Исторически вышло, что на практике модель взаимодействия открытых систем не применяется. Раньше существовали её буквальные реализации, содержащие ровно 7 слоев. Однако со временем их вытеснил менее предписывающий набор протоколов TCP/IP, на котором построен современный Интернет. Тем не менее, модель OSI до сих пор используется в качестве эталона для обучения и документации.


\subsection*{Сети пакетной коммутации. Маршрутизация сообщений.}\label{b17:part2}

\textcolor{olive}{Клик сюда~\ref{b17:part1} к началу билета.}

\textbf{Пакетная коммутации} -- процесс передачи пакетов с использованием транзитных узлов.

\textbf{Маршрутизация} -- процесс определения маршрута данных в сетях связи.

Хотим передать большой файл от A к B. Наше сообщение делится на пакеты, в соответствии с используемым протоколом.
А -> роутер -> маршрутизатор.
У одного маршрутизатора есть несколько физических входящих и исходящих каналов. Соответственно на один маршрутизатор могут приходить пакеты с разных узлов. Внутри маршрутизатора своя очередь приема пакетов и соответственно очередь передачи.
Задача маршрутизатора глядя на логические адреса (куда нужно доставить сообщение), выбрать оптимальный маршрут.

Очередь отправки может переполниться. (Лектор говорит, что очередь приёма не переполняется, так как время обработки пакета в маршрутизаторе согласовано с пропускной способностью.) Может случиться, что пакеты пришли с двух узлов, а отправить их маршрутизатор решил дальше по одному каналу -- в этот момент может произойти переполнение.
Маршрутизатору не хватает ресурсов запомнить все пакеты, которые нужно отправить, а пропускной способности соединения не хватает, чтобы отправить эти пакеты. В такой ситуации маршрутизатор может потерять пакет.

Когда передаем сообщение от А к В, это сообщение проходит через какое-то количество промежуточных устройств. Чем больше устройств, тем больше шанс потерять сообщение, тем больше задержка.

Внутри маршрутизатора строятся таблицы маршрутизации -- они помогают определять оптимальный путь сообщения. Так же в случае недоступности узла, спустя время все пути перестраиваются (для ещё не отправленных сообщений).
\newpage
\section{Билет 18. Невозможность гарантированной доставки сообщений. Алгоритм скользящего окна.} \label{b18:part1}

Перейти к~\nameref{b18:part2}

\textbf{Надёжность} -- доставка без потерь и дублирования. \newline
\noindent A, B -- процесс, программа которая выполняется. \newline
\noindent \textbf{NCP} -- Network Control Procedure, управление сетью.

У нас есть взаимодействие между приложением и NCP, между NCP и сетью передачи данных.
Если мы хотим обеспечить «надёжность» передачи данных (в кавычках, потому что реальную надёжность обеспечить невозможно), то между NCP и сетью передачи данных должен быть протокол взаимодействия.

Приложение А отправляет сообщение и забывает о нём, а NCP A взаимодействует с NCP B по ненадёжной сети передачи данных.
У NCP есть два события: \textbf{recieve} (получение) -- из сети передачи данных в NCP пришло сообщение, и \textbf{deliver} (доставка) -- NCP решило что сообщение может быть доставлено процессу.
\newline
\begin{figure}[H] \centering
	\includegraphics[scale = 1]{18/common.pdf}
\end{figure}

\textbf{\textit{Отличие recieve и deliver}}: в NCP пришёл пакет 2, но не пришел пакет 1 -- событие recieve произошло, но мы не можем отправить пакет 2 в приложение, потому что нарушим порядок -- событие deliver не произошло.

Ненадёжность доставки сводиться к потере пакетов в силу неисправности оборудования или ресурсного ограничения на любом шаге взаимодействия.
\bigskip

Надёжная передача между А и В невозможна, например, если NCP может потерять состояние диалога (перезапуститься).
Пусть NCP B должна доставить процессу B информацию после получения сообщения M от NCP A.
Если NCP B перезапустится после отправки M, то ни один из процессов не может понять было ли доставлено сообщение: у А нет информации, что происходит на стороне B (это удаленная система); так так система со стороны B перезапустилась, то она ничего не помнит.
Повторная передача может привести к дублированию, отказ -- к потере.
\textbf{\textit{Вывод: мы не можем контролировать факт доставки из NCP в процесс.}}

\subsection*{Протоколы, которые передают данные между NCP.}
Рассмотрим разные протоколы общения между NCP. Их можно разделить по количеству сообщений которые передаются.

\textbf{\textit{С 1 сообщением}}: «отправил и забыл» (UDP).

\textbf{\textit{С 2 сообщениями}}: в протокол добавляется подтверждение о получении сообщения -- <\textbf{ack}> (acknowledgment).

\begin{algorithm}
	\caption{Протокол с 2 сообщениями. Нормальный сценарий.}
	\begin{enumerate}
		\item NCP A отправляет <\textbf{data}, m>
		\item NCP B получает <\textbf{data}, m> \\
			доставляет m; отправляет <\textbf{ack}>; закрывает сеанс
		\item NCP A получает <\textbf{ack}> \\
			уведомляет процесс о доставке; закрывает сеанс
	\end{enumerate}
\end{algorithm}

\textbf{DN} (Data network) -- сеть передачи данных.

\begin{algorithm}
	\caption{Протокол с 2 сообщениями. Таймер. Сообщение дублируется.}
	\begin{enumerate}
		\item NCP A отправляет <\textbf{data}, m>
		\item NCP B получает <\textbf{data}, m> \\
			доставляет m; отправляет <\textbf{ack}>; закрывает сеанс
		\item DN Потеря сообщения (<\textbf{ack}>)
		\item NCP A ожидает timeout \\
			убеждается, что к нему не пришло подтверждение и отправляет <\textbf{data}, m>
		\item NCP B получает <\textbf{data}, m> \\
			доставляет m; отправляет <\textbf{ack}>; закрывает сеанс
		\item NCP A получает <\textbf{ack}> \\
			уведомляет процесс о доставке; закрывает сеанс
	\end{enumerate}
\end{algorithm}

\newpage

\textcolor{olive}{Здесь не путать сообщения протокола и те, что отправляем.}\\
В следующем примере, как и в предыдущих, протокол с 2 сообщениями (\textbf{data} + <\textbf{ack}>). И к тому же отправляем 2 сообщения <\textbf{data}, $m_1$> и <\textbf{data}, $m_2$>.

\begin{algorithm}
	\caption{Протокол с 2 сообщениями. Потеря.}
	\begin{enumerate}
		\item NCP A отправляет <\textbf{data}, $m_1$>
		\item NCP B получает <\textbf{data}, $m_1$> \\
			доставляет $m_1$; отправляет <\textbf{ack}>; закрывает сеанс
		\item NCP A ожидает timeout \\
			убеждается, что к нему не пришло подтверждение и отправляет <\textbf{data},  $m_1$>
		\item NCP B получает <\textbf{data},  $m_1$> \\
			доставляет m; отправляет <\textbf{ack}>; закрывает сеанс
		\item NCP A получает <\textbf{ack}> (отправленное на шаге 4) \\
			уведомляет процесс о доставке; закрывает сеанс
		\item NCP A отправляет <\textbf{data}, $m_2$>
		\item DN Потеря сообщения (<\textbf{data}, $m_2$>)
		\item NCP A получает <\textbf{ack}> (отправленное на шаге 2) \\
			уведомляет процесс о доставке; закрывает сеанс
	\end{enumerate}
\end{algorithm}

Получилось, что А получил дважды подтверждение от получении <\textbf{data}, $m_1$>, но с его стороны это выглядело как 2 подтверждения для $m_1$ и $m_2$. Процесс А думает, что передача успешная, а на самом деле B не получил второе сообщение.

\newpage
\textbf{\textit{С 3 сообщениями}}: в протокол хотим добавить подтверждение, что получено подтверждение <\textbf{close}>. Повторная передача при потере <\textbf{data}, m> или <\textbf{ack}>.

\begin{algorithm}
	\caption{Протокол с 3 сообщениями. Нормальный сценарий.}
	\begin{enumerate}
		\item NCP A отправляет <\textbf{data}, m>
		\item NCP B получает <\textbf{data}, m> \\
			доставляет m; отправляет <\textbf{ack}>
		\item NCP A получает <\textbf{ack}> \\
			уведомляет процесс о доставке; отправляет <\textbf{close}>; закрывает сеанс
		\item NCP B получает <\textbf{close}>; закрывает сеанс
	\end{enumerate}
\end{algorithm}

\begin{algorithm}[h!]
	\caption{Протокол с 3 сообщениями. Потеря.}
	\begin{enumerate}
		\item NCP A отправляет <\textbf{data}, $m_1$>
		\item NCP B получает <\textbf{data}, $m_1$> \\
			доставляет $m_1$; отправляет <\textbf{ack}>
		\item NCP A получает <\textbf{ack}> \\
			уведомляет процесс о доставке; отправляет <\textbf{close}>; закрывает сеанс
		\item DN потеря сообщения (<\textbf{close}>)
		\item NCP A отправляет <\textbf{data}, $m_2$>
		\item DN потеря сообщения (<\textbf{data}, $m_2$>)
		\item NCP B ожидает timeout (ждет ответа от A, после шага 2) \\
			убеждается, что к нему не пришло подтверждение  и отправляет <\textbf{ack}>
		\item NCP A получает <\textbf{ack}> \\
			уведомляет процесс о доставке; <\textbf{close}>; закрывает сеанс
		\item NCP B получает <\textbf{close}>; закрывает сеанс
	\end{enumerate}
\end{algorithm}


Получилось, что B отправил повторный <\textbf{ack}>, так как не получил от А  <\textbf{close}>. Процесс А думает, что это были 2 подтверждения на получение сообщений $m_1$ и $m_2$. Хотя на самом деле B не получил второе сообщение.

\subsection*{Алгоритм скользящего окна.}\label{b18:part2}

Перейти к~\nameref{b18:part1}

\href{https://www.youtube.com/watch?v=hd6QNXK5rPk}{Хорошее видео на эту тему.}

Ранее рассмотренные протоколы, после отправки ждут подтверждения. Скользящее окно применяется в протоколе TCP.

Концепция \textbf{скользящего окна} (sliding window) заключается в том, что для повышения скорости передачи данных отправителю разрешается передать некоторое количество сообщений, не дожидаясь прихода на эти пакеты подтверждения.

\begin{figure}[H] \centering
	\includegraphics[scale = 0.35]{18/send_types.png}
	\caption{Схематичное различие ранее рассмотренных протоколов и алгоритма скользящего окна}
\end{figure}

\textbf{Размер скользящего окна} -- количество сообщений, которые могут быть отправлены без подтверждения. \\
В примере будет окно размера 8. Без подтверждения можем отправить 8 сообщений.
\begin{figure}[H] \centering
	\includegraphics[scale = 0.3]{18/window_1.png}
\end{figure}

Получили часть подтверждений на ранее отправленные данные -- сдвигаем окно на количество полученных подтверждений и отправляем новую порцию данных. В примере получили 3 подтверждения, сдвинули окно на 3, отправили ещё 3 сообщения. После этого отправитель ожидает подтверждения.


\begin{figure}[H] \centering
	\includegraphics[scale = 0.4]{18/window_2.png}
	\includegraphics[scale = 0.4]{18/window_3.png}
	\caption{Сдвиг окна}
\end{figure}

\newpage
\section {Билет 19. Многоадресная передача. Протоколы B-, R-, CO-, ТО-multicast: устойчивость к сбоям, сложность по числу сообщений и времени.}
\newpage
\section{Билет 20. Отношение причинно-следственной зависимости. Метки Лэмпорта, векторные часы.}\label{b20}
\begin{center}
    \textit{\underline{Отношение причинно-следственной зависимости.}}
\end{center}
 Отношение причинно-следственной зависимости\footnote{Подробнее см. 66 с. книги «Ж. Тель. Введение в распределенные алгоритмы. М.: МЦНМО, 2009. — 616 с. — ISBN 978-5-94057-515-3.»} - это аналог понятия  «Справедливости». Можно сравнивать события и узнавать, какое из них произошло раньше, а какое позже. \\ 
Пусть в предположении у нас задано следующее:
\begin{itemize}
\item Сравнимы события в одном процессе (например, с помощью локального счётчика).
\item Сравнимы отправка и получение сообщений. (Отправка сообщения логически предшествует получению этого же самого сообщения.)
\end{itemize}
%Представляя выполнение в виде последовательности переходов, мы тем самым естественно привносим в модель понятие времени. Будем говорить, что переход $a$ происходит раньше, чем переход $b$, если в последовательности переходов событие $a$ предшествует событию $b$.\\
Хотим реализовать механизм сравнения двух событий в случае, если они сравнимы.\footnote{Узнать что произошло раньше, а что позже (транзитивное замыкание описанных двух отношений).}

Пусть $n$ - количество процессов. $\forall p = 1, \ldots, n$ заводим свой счётчик $L_p := 0;$ событий.\\
Внутри процесса могут происходить локальные события, которые мы хотим учесть при общем сравнении, поэтому мы их тоже будем помечать. (см. \nameref{algComparison}).
\begin{algorithm}
\caption{Алгоритм сравнения событий. Метки Лэмпорта}
\label{algComparison}
\begin{algorithmic}
\State $L_p \gets 0$
\Function{matching}{$e$}\Comment{Приписываем событию $e$ временную метку $L(e)$ (Leslie Lamport).}
%\State 
\If{$e$ - локальное событие в процессе $p$}
    \State $L_p += 1$
    \State $L(e) \gets L_p$ \Comment{Нумеруем локальное событие своим же счётчиком.}
\EndIf
\If{$e$ - отправка сообщения $message$ процессу $q$ от процесса $p$}
    \State $L_p += 1$
    \State $send(<message, L_p>, q)$ \Comment{Вместе с данными передаём свою метку времени.}
\EndIf
\If{$e$ - получение сообщения $<message, L>$ от процесса $q$ в  процессе $p$}
    \State $L_p \gets \max\{L,L_p\} + 1$
    \State $L(e) \gets L_p$ \Comment{Получение сообщения тоже может быть учтено как событие $e$.}
\EndIf
\EndFunction
\end{algorithmic}
\end{algorithm}
%Приписываем событию $e$ временную метку $L(e)$ (Leslie Lamport).\\ \\
%Когда происходит локальное событие $e$ в процессе с номером $p$:\\
%\ \ \ \ $L_p += 1;$\\
%\ \ \ \ $L(e) = L_p;$ //Нумеруем локальное событие своим же счётчиком. \\ \\ 
%Отправка сообщения $message$ процессу с номером $q$ от процесса с номером $p$:\\
%\ \ \ \ $L_p += 1;$\\
%\ \ \ \ $send(<message, L_p>, q);$ //Вместе с передаваемыми данными передаём свою метку времени.\\ \\
%Получение сообщения $<message, L>$ от процесса с номером $q$ в  процессе с номером $p$:\\ 
%\ \ \ \ $L_p = \max\{L,L_p\} + 1;$\\
%\ \ \ \ $L(e) = L_p;$ //Получение сообщения тоже может быть учтено как событие $e$. \\
Кратко поясним работу данного алгоритма. У всех процессов исходно нулевой счётчик. Каждому событию, которое мы хотим сравнивать, приписываем числовую временную метку. Для локальных событий в $p$ увеличиваем локальный счётчик $L_p$ на $1$. Отправляя сообщение, локально подписываем факт отправки как отдельное событие и затем отправляем свою метку. Получая сообщение, вычисляем максимум по временным меткам.
\begin{theorem}
Пусть событие $e_1$ произошло точно раньше события $e_2$ по заданному отношению причинно-следственной зависимости (либо произошли в одном процессе и $e_1$ раньше $e_2$, либо в разных, но между ними $\exists$ цепочка попарно-сравнимых событий).
Тогда $L(e_1) < L(e_2)$.
\end{theorem}
\textbf{\hypertarget{backto}{Замечание:}} Обратное неверно! (См. рисунок. \ref{fig:image_Comparsion})
\begin{figure}[h!]
\center{\includegraphics[width=0.9\textwidth]{20/Comparison_for_labels.jpg}}
\caption{В момент $t_0$: $2 = L_{p_2} < L_{p_1} = 3$, хотя событие $e_1 < e_2$. \protect\hyperlink{backto}{$\circlearrowleft$}}
\label{fig:image_Comparsion}
\end{figure}
        %\includegraphics[width=0.5\textwidth]{Comparison_for_labels.jpg}
        
Для выполнения необходимого и достаточного условия ($e_1 < e_2 \Leftrightarrow L(e_1) < L(e_2)$) реализуют механизм под названием «векторные часы». Разница с предыдущим алгоритмом лишь в том, что во всех процессах ведётся учёт временных меток не одного процесса, а целого вектора (массива) временных меток $\bar{L}_p := (L_p^1, \ldots, L_p^n)^T$ всех процессов.
Приведём \nameref{algComparisonVecClock}.
\begin{algorithm}
\caption{Алгоритм сравнения событий. Векторные часы}
\label{algComparisonVecClock}
\begin{algorithmic}
\State $L_p \gets (\underbrace{0,\ldots,0}_{n}) = \bar{0}$
\Function{matching}{$e$} \Comment{Приписываем событию $e$ временную метку $L(e)$ (Leslie Lamport).}
\If{$e$ - локальное событие в процессе $p$}
    \State $L_p[p] += 1$
    \State $L(e) \gets L_p$ \Comment{Нумеруем локальное событие своим же счётчиком.}
\EndIf
\If{$e$ - отправка сообщения $message$ процессу $q$ от процесса $p$}
    \State $L_p[p] += 1$
    \State $send(<message, L_p>, q)$ \Comment{Вместе с данными передаём свою метку времени.}
\EndIf
\If{$e$ - получение сообщения $<message, L>$ от процесса $q$ в  процессе $p$}
    \For{$i = 1, \ldots, n$}
        \State $L_p[i] \gets \max\{L[i],L_p[i]\}$
    \EndFor
    \State $L_p[p] += 1$
    \State $L(e) \gets L_p$\Comment{Получение сообщения тоже может быть учтено как событие $e$.}
\EndIf
\EndFunction
\end{algorithmic}
\end{algorithm}
%$\forall p = 1, \ldots, n$ заводим свой массив счётчиков $\bar{L}_p := \bar{0};$ событий.\\ \\
%Когда происходит локальное событие $e$ в процессе с номером $p$:\\
%\ \ \ \ $L_p[p] += 1;$\\
%\ \ \ \ $L(e) = L_p;$ //Нумеруем локальное событие своим же счётчиком. \\ \\
%Отправка сообщения $message$ процессу с номером $q$ от процесса с номером $p$:\\
%\ \ \ \ $L_p[p] += 1;$\\
%\ \ \ \ $send(<message, L_p>, q);$ //Вместе с передаваемыми данными передаём свою метку времени.\\ \\
%Получение сообщения $<message, L>$ от процесса с номером $q$ в  процессе с номером $p$:\\
%\ \ \ \ $L_p[i] = \max\{L[i],L_p[i]\};\ \forall i = 1, \ldots, n.$\\
%\ \ \ \ $L_p[p] += 1;$\\
%\ \ \ \ $L(e) = L_p;$ //Получение сообщения тоже может быть учтено как событие $e$.\\
Для этого алгоритма необходимое и достаточное условие выполняется в следующем виде.
\begin{theorem}
$e_1 < e_2 \Leftrightarrow L(e_1)[i] \leq L(e_2)[i]\ \forall i = 1,\ldots, n$ покоординатно, причём $\exists i_0$ такое, что неравенство $ L(e_1)[i_0] < L(e_2)[i_0]$ строгое.
\end{theorem}
\textbf{Замечание 1:} 
Если $\exists i_1, i_2 \in [1,n]: L(e_1)[i_1]< L(e_2)[i_1]$ и при этом же $L(e_1)[i_2] >  L(e_2)[i_2] $, то события $e_1$ и $e_2$ не сравнимы.
\\ 
\textbf{Замечание 2:} Multicast, построенный на основе векторных часов - это \nameref{b19:part5}. Не можем доставить пришедшее сообщение в процесс до тех пор, пока не получили все сообщения, логически предшествующие тому, что стоит в  очереди.

\newpage
\section {Билет 21. Алгоритмы избрания лидера. Свойства завершения, единственности и согласия. Выборы в кольцевой сети. Выборы на основе протоколов многоадресной передачи. Справедливость.}
\newpage
\section {Билет 22. Взаимное исключение. Решение на основе сервера. Алгоритм Дейкстры с общими регистрами. Алгоритм Петерсона для двух процессов. Алгоритм на основе голосования (Maekawa).}
\textbf{Взаимное исключение}.

Постановка задачи: \\
Задано N процессов, конкурирующих за \textbf{общий} ресурс. Каждый процесс циклически выполняет код, принадлежащий \textbf{критической секции} (CS), и не принадлежащий ей (т.е. иногда процесс заходит в критическую секцию и потом из неё выходит).
Задача стоит в разработке алгоритма, который гарантирует, что в каждый момент времени в критической секции находится не более одного процесса.

Требования: \\
Основные:
\begin{itemize}
\item \textbf{Safety}: Не более одного процесса находится в критической секции
\item \textbf{Liveness}: Если один или несколько процессов хотят зайти в CS, то рано или поздно по крайней мере один из них зайдет в критическую секцию (при условии, что время нахождения процессов в CS ограничено)
\end{itemize}
Дополнительные:
\begin{itemize}
\item \textbf{Отсутствие голодания}: Если процесс p хочет войти в критическую секцию, то рано или поздно он войдет в критическую секцию \footnote{\textit{прим. авт.:} из Liveness это условие не следует} 
\item \textbf{Справедливость}: Если процесс p захотел зайти в CS раньше процесса q, то он зайдет в неё раньше
\end{itemize}

Задача ставится не только для систем с распределенной памятью, но и для систем с общей памятью: процессы могут взаимодействовать через общие переменные; атомарны только операции чтения или записи.

\begin{algorithm}
\caption{Алгоритм Деккера}
Решение задачи взаимного исключения для двух процессов: 
\label{algDekker}
\begin{algorithmic}
\Ensure \\$status[2]$ \Comment{массив из двух чисел. Для чтения каждому из процессов. Для записи у $i$-ого процесса есть доступ только к $i$ ячейке.}\\ 
$turn$ \Comment{переменная, в которую оба процесса могут записывать и читать данные}\\ 
\While{$status[other] == competing$} 
  \If{$turn == other$} 
    \State $status[i] = out$
    \State $wait until (turn == i)$
    \State $status[i] = competing$
  \EndIf
\EndWhile

\State \textbf{CS}
\State $turn = other$
\State $status[i] = out$
\end{algorithmic}
\end{algorithm}


\begin{algorithm}
\caption{Алгоритм Дейкстры}
Решение задачи взаимного исключения для N процессов: 
\label{algDijkstra}
\begin{algorithmic}
\Ensure \\$status[i] \in \{competing, out, cs\}$ \\
$turn \in \{1,...N\}$ 

\Do
  \While{$turn \neq i$} 

    \If{$status[turn] == out$} 
      \State $turn := i$
    \EndIf
  \EndWhile
  \State $status[i] = cs$
\doWhile{$\exists other:\ status[other] = cs$}
\State \textbf{CS}
\State $status[i] = out$
\end{algorithmic}
\end{algorithm}

\begin{algorithm}
\caption{Алгоритм Петерсона}
Решение задачи взаимного исключения для 2 процессов: 
\label{algPeterson}
\begin{algorithmic}
\State Перед тем как начать исполнение критической секции кода, процесс должен вызвать процедуру LOCK() со своим номером в качестве параметра. Она должна организовать ожидание процессом своей очереди входа в критическую секцию. После исполнения критической секции и выхода из неё процесс вызывает другую процедуру UNLOCK(), после чего уже другой процесс сможет войти в критическую область. 
\Ensure \\$want[2]$ \Comment{массив из двух чисел}\\ 
$victim$ \Comment{переменная, в которую оба процесса могут записывать и читать данные}\\ 
\Procedure{lock:}{}
  \State $want[i] = true$  
  \State $victim = i$
  \While{$want[1-i] == true$ \textbf{and} $victim == i$}
    \State pass
  \EndWhile
\EndProcedure

\\
\Procedure{unlock:}{}
  \State $want[i] = false$
\EndProcedure
\end{algorithmic}
\end{algorithm}


\nameref{algDijkstra} не удовлетворяет требованию справедливости (нужный процесс зайдет в критическую секцию, а остальные могут зациклиться в алгоритме)

\newpage
\section{Билет 23. Согласие в распределенной системе. Задача византийских генералов: распространение значения (agreement), согласование решения (consensus), согласование вектора (interactive consistency). Невозможность решения при трёх процессах и одном сбое. Алгоритм Лэмпорта для "устных" сообщений. Пример распределенных транзакций.}
\newpage
\section{Билет 24. Отказоустойчивость. Активная и пассивная репликация. Алгоритмы поддержания согласованного состояния реплик. Построение надежного хранилища из ненадежных компонентов.}

\begin{center}
	\textit{\underline{Отказоустойчивость}}
\end{center}

\textbf{Отказоустойчивость} — свойство технической системы сохранять свою работоспособность после отказа одной или нескольких её составных частей. Отказоустойчивость определяется количеством единичных отказов составных частей (элементов) системы, после наступления которых сохраняется работоспособность системы в целом. Базовый уровень отказоустойчивости подразумевает защиту от отказа одного любого элемента, однако в теории можно реализовать систему с практическим любым количеством отказов.

Примером отказоустойчивой системы является \href{https://ru.wikipedia.org/wiki/RAID}{RAID}.


\begin{center}
	\textit{\underline{Активная и пассивная репликации}}
\end{center}

В построении отказоустойчивых систем есть два главных подхода к проектированию: активная и пассивная репликация.

\textbf{Активная репликация (на англ. State Machine Approach)} - каждый элемент (реплика) системы хранит копию состояния данных. Операции чтения и модификации выполняются локально в каждом узле. Для поддержания когерентности реплик операции модификации рассылаются всем репликам объекта, которые выполняют их над локальной копией состояния. При выходе из строя части реплик проводится <<голосование>> для выбора реплики, которая будет общаться с клиентом как единственный сервер, поэтому для клиента извне вся система выглядит как один сервер. <<Голосование>> осуществляется с помощью следующих алгоритмов: \href{https://clck.ru/JVS7z}{протокол «Византийского соглашения»}, \nameref{b19:part2}, \href{https://clck.ru/q62LJ}{алгоритм консенсуса}). Отсутствует централизованный контроль, однако в процессе работы выполняется много избыточных вычислений и коммуникаций.
\begin{figure}[H]
	\centering
	\includegraphics[scale = 0.7]{24/active.png}
	\caption{Схема активной репликации}
	\label{fig:active_repl}
\end{figure}

\textbf{Пассивная репликация (на англ. Primary-Backup Approach)} - каждый элемент (реплика) системы хранит копию состояния данных. Одна из реплик назначается главной. Операции чтения выполняются локально во всех узлах. Операции, модифицирующие состояние объекта, направляются главной реплике, которая, после выполнения метода, обновляет все остальные реплики. При выводе из строя главной реплики пользователь замечает проблемы с доступом к данным, до тех пор пока одна из запасных реплик не возьмет на себя роль главной. Данный метод обладает намного меньшим потреблением ресурсов и отсутствие большого количества избыточных коммуникаций и вычислений в сравнении с активной репликацией, и поэтому используется чаще.
\begin{figure}[H]
	\centering
	\includegraphics[scale = 0.7]{24/passive.png}
	\caption{Схема пассивной репликации}
	\label{fig:passive_repl}
\end{figure}



	

\newpage
\section {Билет 25. Модели логического разграничения доступа: мандатная, дискреционная, ролевая. Атрибутивная модель.}
\newpage
\section {Билет 26. Конфиденциальные вычисления. Гомоморфное шифрование и многосторонние вычисления.}

\textbf{Конфиденциальные вычисления} используются в тех случаях, когда необходимо обеспечить защиту некой информации ввиду из-за законодательных запреты, коммерческой тайны и т.д. При этом:
\begin{itemize}
\item некоторые вычисления можно провести только в <<недоверенной среде>> (
например непроверенное оборудование или ПО),
\item некоторые вычисления требуют слияния данных сразу нескольких клиентов, соотвественно требуется обеспечить конфиденциальность данных между разными клиентами. 
\end{itemize}

\textbf{Примеры, требующие использования конфиденциальных вычислений}:
\begin{itemize}
	\item Осуществление некоторых вычислений в <<облаке>>.
	\item Конфиденциальный поиск. Выполнение поисковых запросов в такой форме, чтобы БД (например Google) не знал, какая информация ищется.
	\item Машинное обучение на совместных данных.
	\item Вычисление значения функций от секретных функций. Например <<задача о двух миллионерах>>\url{https://clck.ru/pq7Zm}
\end{itemize}

Для каждой из задач может быть реализован специальный алгоритм, реализующий обеспечивающий \textbf{конфиденциальность вычислений}. В качестве примера возьмем задачу о  машинном обучении на совместных данных, ее конфиденциальный вариант называется \textbf{федеративное обучение}. Если вкратце, то используется одна модель для обучения данных, но все данные остаются у их владельца. Например подсказки при вводе на клавиатуре телефонов на Android: модель используется одна, однако все данные и их обработка хранятся и осуществляются на телефоне пользователя, а не где-то на серверах Google. На сервера отправляется только часть данных, по которым нельзя узнать что вводил конкретный человек на конкретном телефоне.(на лекциях это не особо обсуждалось, подробнее почитать можно вот здесь \url{https://www.machinelearningmastery.ru/federated-learning-a-new-ai-business-model-ec6b4141b1bf/})

Для произвольной задачи обычно используются следующие подходы:
\begin{itemize}
	\item Гомоморфное шифрование
	\item Многосторонние вычисления
	\item Функциональное шифрование 
\end{itemize}

Рассмотрим подробнее каждый из подходов.
\subsection {Функциональное шифрование}
Так, в схеме \textbf{функционального шифрования} для функции $F(\cdot, \cdot)$ шифрователь с мастер-ключом генерирует ключ $s_k$, позволяющий вычислять функцию $F(k, \cdot)$ от зашифрованных данных так, что расшифровщик, зная зашифрованный текст $c$ от данных $x$ и ключа $s_k$, был способен вычислить $F(k, x)$, не имея возможности узнать что-либо кроме результата вычисления функции по $x$. Лектор сказал что особо не понимает как это все работает и не знает реальных примеров использования, так что если интересно то поподробнее можно почитать здесь: \url{https://habr.com/ru/post/482790/}, \url{https://clck.ru/pqCRB}.


\subsection{Гомоморфное шифрование}
\textbf{Гомоморфным шифрованием} называется раздел криптографии, который посвящен разработке алгоритмов допускающих выполнение вычислений над \textbf{шифротекстами}. В качестве криптографических протоколов используются как симметричные (с одним ключом), так и асимметричные (с открытым и закрытым ключом). \url{https://otus.ru/nest/post/726/}

В \textbf{гомоморфном шифровании} подразумевается, что мы можем производить вычисления от \textbf{шифротекстами} и что для произвольной \textbf{вычислимой} функции $f$ верно (Enc = Encryption = шифрование):
$$Enc(f(x, y)) = f(Enc(x), Enc(y))$$

Под \textbf{вычислимыми} функциями подразумеваются функции, которые можно задать с помощью схемы из элементов на Рис. \ref{fig:logicscheme} (Что в целом эквивалентно определениям из всяких Матлогов/ТДФ/Дискры и т.д)
\begin{figure}[H]
	\centering
	\includegraphics[scale = 0.3]{26/logic_cheme.jpeg}
	\caption{Логические элементы}
	\label{fig:logicscheme}
\end{figure}

Криптографическая схема состоит из 4х компонент:
\begin{enumerate}
	\item KeyGen - алгоритм создания открытого(PublicKey, $pk$) и закрытого (SecretKey, $sk$) ключей (RSA, DSA, Elgamal, и т.д.)
	\item Encryptioner - алгоритм шифрования, который по $pk$ создает \textbf{шифротекст} $c$ от исходных данных $x$: $c = Enc(pk, x)$
	\item Decryptioner - алгоритм дешифрования, который по $sk$ расшифровывает \textbf{шифротекст} $c$ от исходных данных $x$: $x = Dec(sk, c)$
	\item Evaluator - алгоритм, позволяющий вычислять булевы схемы.
\end{enumerate} 

Для булевой схемы $f$ с $k$ входами, произвольной пары $(pk, sk)$ и любого $(x_1, \dots, x_k) \in \{0;1\}^k$ должно выполняться следующее: $$Dec(sk, Eval(pk, f, c_1,c_2, \dots, c_k)) = f(x_1, \dots, x_k), \, c_i = Enc(pk, x_i)$$


Взломостойкость данного метода зависит от KeyGen, Enc и Dec. В настоящее время безопасность подобных методов с открытыми ключами строиться на вычислительной сложности разложении числа на простые множители. Пусть выбрано число $p \in \mathbb{Z}$ и случайные числа $q_1, \dots, q_n \in \mathbb{N}$. Вычислим значения $m_i=pq_i$. При достаточно большом $n$ можно эффективно восстановить $p$ по $m_i$: вычислив $gcd(m_1, \dots, m_n)$.  Однако данная задача легко решается с помощью квантовых компьютеров. Поэтому используется так называемый алгоритм \textbf{обучения с ошибками}. Считается, что если вместо точных значений $m_i$ известны значения $pq_i+r_i$, где $r_i \in \mathbb{Z}$, то такая задач является вычислительно сложной даже для квантового компьютера. На данном приеме строятся схемы \textbf{гомоморфного шифрования}.

Cхема простого \textbf{гомеоморфного шифрования}. В качестве секретного ключа $sk$ возьмем число $p \in \mathbb{N}$. Чем больше $p$ тем сложнее будет взломать данную схему.
\begin{itemize}
	\item Enc: для получения \textbf{шифротекста} $c_i$ с одного бита $b_i \in \{0, 1\}$ исходного сообщения вычислим $c_i = pq_i + 2r_i + b_i$, где $q_i$ и $r_i$ -- случайные числа, которые с равной вероятностью выбираются из определенных диапазонов.
	\item Dec: для расшифровывания \textbf{шифротекста} $c_i$ сначала вычисляется $c_i^{'} = (c_i \, mod \, p) \in (-p/2, p/2)$, а затем вычисляем $c_i \, mod \, 2$. Т.е $b_i = ((c_i \, mod \, p) \, mod \, 2)$. Под $a \, mod \, b $ подразумевается остаток от деления $a$ на $b$.
\end{itemize}


Если выполняется неравенство $|2r_i+b_i| < p/2$ (1), то $(c_i mod p) = 2r_i + b_i$, откуда следует корректность формулы для получения открытого текста.

Пусть заданы два \textbf{шифротекста} $c_1 = pq_1+2r_1+b_1, c_2 = pq_2+2r_2+b_2$, тогда их сумму $c = c_1+c_2 = pq_1+2r_1+b_1 + pq_2+2r_2+b_2$ можно представить в виде $p(q_1+q_2) + 2(r_1+r_2) + (b_1 + b_2)$, т.е $c$ является \textbf{шифротекстом} для $b_1 + b_2 \, mod \, 2$ с шумом $r_1 + r_2$. Аналогичным образом представляется произведение двух \textbf{шифротекстов}: $c = c_1 \cdot c_2 = (pq_1+2r_1+b_1)(pq_2+2r_2+b_2)$ с шумом $2r_1r_2 + b_1r_2 + b_2r_2$.

Как можно заметить при операции сложении шум растет медленнее чем при операции умножения, соотвественно при большом количестве операций и росте шума перестает выполняться (1) из-за чего дешифровка результата работы функции становиться невозможной. Решением данной проблемы является метод \textbf{самокорректировка (bootstrapping)}.  \textcolor{olive} {Объяснение как это работает на лекциях было очень странным не думаю что можно это адекватно описать}. Хоть какое-то объяснение кажется есть в этой статье \url{https://eprint.iacr.org/2021/091.pdf}.

\textbf{Примеры схем:}
\begin{itemize}
\item \textbf{BGV (Brakerski-Gentry-Vaikuntanathan)} - Схема ограниченного гомоморфного шифрования, позволяющая выполнять требуемое число арифметических операций над целыми числами без самокоррекции. 
\item \textbf{CKKS (Cheon, Kim, Kim, Song)} - Схема разработана для приближенных вычислений с комплексными числами.
\item \textbf{TFHE (Fast Fully Homomorphic Encryption over the Torus)} - Схема полного гомоморфного шифрования, которая реализует комбинацию логического элемента NAND (Рис. \ref{fig:logicscheme}) с последующей самокорректировкой. 
\end{itemize}

Быстродействие данных методов значительно хуже чем у методов с открытыми данными, примерно в $10^4-10^6$ раз медленнее. Сложение двух 32-битных значений методом \textbf{TFHE} занимает около 1 минуты, умножение 1.5 минуты, а целочисленное деление около 15 минут. Также стоит отметить, что не существует (и в теории не может существовать) <<гомоморфного>> языка программирования, так как невозможно реализовать даже такие базовые синтаксические конструкции как циклы и условные ветвления. 

\subsection{Многосторонние вычисления}
\textbf{Постановка задачи многосторонние вычислений}: участвуют $N$ участников $p_1, p_2, \dots, p_N$. У каждого участника есть тайные входные данные $x_1, x2, \dots, x_N$ соответственно. Участники хотят найти значение $f(d_1, d_2, \dots, d_N)$, где $f$ — известная всем участникам вычислимая функция от $N$ аргументов. Допускается, что среди участников будут получестные нарушители, то есть те, которые верно следуют протоколу, но пытаются получить дополнительную информацию из любых промежуточных данных. Требуется сохранить конфиденциальность входных параметров $x_i$. 

Далее подробно рассмотрим случаи для двух участников ($N=2$).
Тут можно использовать протокол \textbf{Oblivious Transfer (Забывчивая передача, Подробная реализация описана в вики \url{https://clck.ru/puQ3n})}, в котором отправитель передает по одной возможные части информации получателю, но не запоминает (является забывчивым), какие части были переданы, если вообще были. Отправитель имеет два значения $m_0, m_1$. Получатель выбирает $i \in \{0,1 \}$ и запрашивает у отправителя $m_i$ данные, при условии, что:
\begin{itemize}
	\item Получатель не получит данные $m_i-1$. То есть, если запрашиваем $m_0$, то точно не получим $m_1$ и наоборот.
	\item Отправитель не знает $i$, т.е. не знает что он отправил. 
\end{itemize}

\textbf{Oblivious Transfer} используется в одном из основных протоколов многосторонних вычислений -- \textbf{Garbled circuit
 (Искаженная схема) \url{https://ru.wikibrief.org/wiki/Garbled_circuit}, \url{https://en.wikipedia.org/wiki/Garbled_circuit}}. Опишем протокол:
\begin{enumerate}
	\item Базовая функция (например, у миллионеров) задача, функция сравнения) описывается как логическая схема с двумя входами. Схема известна обеим сторонам. Этот шаг может быть выполнен сторонним лицом.
	\item Алиса искажает (шифрует) схему. Мы называем Алису манипулятором.
	\item Алиса отправляет искаженную схему Бобу вместе со своим зашифрованным вводом.
	\item Для того, чтобы вычислить схему, Боб также должен искажать свой собственный ввод. Для этого ему нужна помощь Алисы. Потому что только мусорщик умеет зашифровать. Наконец, Боб может зашифровать свой ввод, не обращая внимания на передачу. В терминах определения, приведенного выше, Боб является получателем, а Алиса - отправителем при этой не обращающей внимания передаче. Именно в этом шаге используется \textbf{Oblivious Transfer}
	\item Боб оценивает (расшифровывает) схему и получает зашифрованные выходные данные. Мы называем Боба оценщиком.
	\item Алиса и Боб обмениваются данными для изучения вывода.
\end{enumerate}



\newpage
\begin{thebibliography}{100}
	\bibitem{Zhel} Введение в распределённые алгоритмы,
    \emph{Ж. Тель, 2009}
	
\end{thebibliography}

\end{document}
